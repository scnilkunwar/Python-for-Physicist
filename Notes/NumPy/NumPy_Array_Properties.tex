\documentclass[a4paper, 12pt]{article}
\usepackage{amsmath}
\usepackage{geometry}
\geometry{a4paper, margin=1in}
\usepackage{listings}
\usepackage{xcolor}

\lstset{
    language=Python,
    basicstyle=\ttfamily\footnotesize,       % Font style for code
    keywordstyle=\bfseries\color{teal},      % Style for keywords
    stringstyle=\color{orange},              % Style for strings
    commentstyle=\color{green!50!black}\itshape, % Style for comments
    numbers=left,                            % Line numbering
    numberstyle=\tiny\color{gray},           % Style for line numbers
    stepnumber=1,                            % Increment of line numbers
    numbersep=8pt,                           % Space between numbers and code
    frame=single,                            % Frame around code
    rulecolor=\color{black},                 % Frame color
    backgroundcolor=\color{white!97!gray},   % Code background color
    breaklines=true,                         % Automatic line breaks
    captionpos=b,                            % Position of captions
    showstringspaces=false,                  % Don't show spaces in strings
    tabsize=4,                               % Tab size
    morekeywords={as, assert, async, await}, % Additional Python keywords
    xleftmargin=0.5cm                        % Left margin space
}

\title{NumPy Array Properties}
\author{Sunil Kunwar}
\date{\today}

\begin{document}

\maketitle

\section{Introduction}
NumPy arrays come with several important properties that help in understanding and manipulating them effectively. Below are the key properties of a NumPy array along with examples.

\section{Properties of NumPy Arrays}

\subsection{ndim (Number of Dimensions)}
This property returns the number of dimensions (or axes) of the array.
\begin{lstlisting}
arr = np.array([[1, 2, 3], [4, 5, 6]])
print(arr.ndim)  # Output: 2
\end{lstlisting}

\subsection{shape (Shape of the Array)}
Returns a tuple representing the shape of the array.
\begin{lstlisting}
print(arr.shape)  # Output: (2, 3)
\end{lstlisting}

\subsection{size (Total Number of Elements)}
Returns the total number of elements in the array.
\begin{lstlisting}
print(arr.size)  # Output: 6
\end{lstlisting}

\subsection{dtype (Data Type of Elements)}
Returns the data type of the elements in the array.
\begin{lstlisting}
print(arr.dtype)  # Output: dtype('int64')
\end{lstlisting}

\subsection{itemsize (Size of Each Element in Bytes)}
Returns the size of each element in the array, in bytes.
\begin{lstlisting}
print(arr.itemsize)  # Output: 8
\end{lstlisting}

\subsection{nbytes (Total Bytes Consumed by the Array)}
Returns the total number of bytes used by the array.
\begin{lstlisting}
print(arr.nbytes)  # Output: 48
\end{lstlisting}

\subsection{T (Transpose of the Array)}
Returns the transpose of the array.
\begin{lstlisting}
print(arr.T)  # Output: [[1, 4], [2, 5], [3, 6]]
\end{lstlisting}

\subsection{flat (Flat Iterator)}
Provides an iterator to iterate over all elements of the array as if it were 1D.
\begin{lstlisting}
for item in arr.flat:
    print(item)  # Output: 1 2 3 4 5 6
\end{lstlisting}

\subsection{real and imag (Real and Imaginary Parts)}
For arrays with complex numbers, these return the real and imaginary parts of each element.
\begin{lstlisting}
complex_arr = np.array([1+2j, 3+4j])
print(complex_arr.real)  # Output: [1. 3.]
print(complex_arr.imag)  # Output: [2. 4.]
\end{lstlisting}

\subsection{base (Base Object if the Array is a View)}
Returns the original array if the array is a view of another array.
\begin{lstlisting}
arr_view = arr[0:2, :]
print(arr_view.base is arr)  # Output: True
\end{lstlisting}

\subsection{strides (Tuple of Bytes to Step in Each Dimension)}
Returns the number of bytes that need to be stepped to move to the next element along each dimension.
\begin{lstlisting}
print(arr.strides)  # Output: (24, 8)
\end{lstlisting}

\subsection{ctypes (Interface to C-types)}
Provides a pointer to the array data for interfacing with C or C++ code.
\begin{lstlisting}
print(arr.ctypes.data)  # Output: memory address of the array
\end{lstlisting}

\subsection{flags (Memory Layout Information)}
Provides detailed information about the memory layout of the array.
\begin{lstlisting}
print(arr.flags)
# Output:
#   C_CONTIGUOUS : True
#   F_CONTIGUOUS : False
#   OWNDATA : True
#   WRITEABLE : True
#   ALIGNED : True
#   UPDATEIFCOPY : False
\end{lstlisting}

\subsection{copy (Copy of the Array)}
Creates a copy of the array.
\begin{lstlisting}
arr_copy = arr.copy()
print(arr_copy)
\end{lstlisting}

\subsection{view (View of the Array)}
Creates a view (or shallow copy) of the array.
\begin{lstlisting}
arr_view = arr.view()
print(arr_view)
\end{lstlisting}

\section{Example: Displaying All Properties of an Array}
Here is a complete example that demonstrates all the properties discussed above:
\begin{lstlisting}
import numpy as np

arr = np.array([[1, 2, 3], [4, 5, 6]])

print("Array:", arr)
print("ndim:", arr.ndim)
print("shape:", arr.shape)
print("size:", arr.size)
print("dtype:", arr.dtype)
print("itemsize:", arr.itemsize)
print("nbytes:", arr.nbytes)
print("T (transpose):", arr.T)
print("flags:", arr.flags)
\end{lstlisting}

\end{document}
